% \batchmode

\hoffset=-0.5in
\voffset=-1in

\font\standard="Arial" at 10pt
\font\standardSmall="Arial" at 8pt
\font\sykl="Wingdings 3" at 11pt
\font\sygr="Wingdings 3" at 13pt


\vsize = 220mm
\hsize = 150mm
\parindent = 0pt

\count0 = 1

\def\bild #1{%
	\hbox to0pt{\vbox to0pt{%
	\special{pdf:image scale 0.25 (#1)}
	\vss}\hss%
}}

\def\item #1\par{\setbox1=\hbox{\sykl\char182\hskip3dd}\setbox2=\hbox{\hskip\wd1\sygr\char234\hskip3dd}
	\hangindent=\wd2\hangafter=1\leavevmode\box2 #1 \par}

\def\subCat #1\par{\setbox1=\hbox{\sykl\char182\hskip3dd}
	\hangindent=\wd1\hangafter=1\leavevmode\box1 #1 \par}

\def\kiste #1#2 {%
	\vbox{\hrule\hbox{\hbox to 0pt{\setbox1=\vbox{\vskip3pt#1\vskip3pt}\dimen1=\hsize\advance\dimen1 by 6pt\special{.9 setgray}\vrule width\dimen1 height\ht1\special{0 setgray}\hss}\vrule\kern3pt\vbox{\vskip3pt#1\vskip3pt}\kern3pt\vrule}\hrule\hbox{\vrule\kern3pt\vbox{\kern3pt#2\kern3pt}\kern3pt\vrule}\hrule\vskip.5cm}
}

\def\wochentrenner #1#2 {{    % Das zweite Paar Klammern macht die Boxen lokal
	\vskip.6cm%
	\dimen2 = 5pt                 % Abstand zwischen Linie und Inhalt
	\setbox1=\hbox{Jahr #1 \raise3.1pt\hbox{\vrule width10pt height.4pt} Kalenderwoche #2}
	\dimen1=\hsize
	\advance\dimen1 by 6pt        % Gleicht die zus�tzliche Breite aus, die die Kisten durch die Abst�nde bekommen
	\advance\dimen1 by 0.8pt      % Gleicht die zus�tzliche Breite aus, die die Kisten durch die Liniendicken bekommen
	\advance\dimen1 by -\wd1
	\dimen1 = .5\dimen1
	\advance\dimen1 by -\dimen2
	\hbox{\raise3.1pt\hbox{\vrule width\dimen1 height.4pt}\hskip\dimen2 \copy1 \hskip\dimen2 \raise3.1pt\hbox{\vrule width\dimen1 height.4pt}}
	\vskip.3cm%
}}

\output={%
	\shipout\vbox to 297mm{%
		\vskip 15mm%
		\hbox to 210mm{%
			\hskip 25mm%
			\vbox to 257mm{%
				\hbox to 150mm{\vbox to 1.59cm{\vfill\hbox to 4.62cm{\bild{creations_fuer_wochenberichte.jpg}\hfill}}\hfill\vbox to 1.59cm{\vskip11pt\hbox{\hbox to 12mm{Name:\hfill}___trainee___}\hbox{\hbox to 12mm{Firma:\hfill}___company___}\hbox{\hbox to 12mm{Seite:\hfill}\the\count0}\vfill}}%
				\unvbox255%
				\vfill%
				\hbox to 160mm{%
					\hfill%
					\vbox{%
						\hsize=60mm%
						\hbox{\standardSmall\signDate}%
						\vskip 1mm%
						\hrule depth 0.2pt height 0.2pt%
						\vskip 2mm%
						\centerline{___trainee___}%
					}%
					\hfill%
					\vbox{
						\hsize=60mm%
						\hbox{\standardSmall\signDate}%
						\vskip 1mm
						\hrule depth 0.2pt height 0.2pt%
						\vskip 2mm
						\centerline{___instructor___}
					}%
					\hfill%
				}%
			}%
			\hfill%
		}%
		\vfill%
	}%
	\global\advance\count0 by 1 % Automatisch funktioniert das wohl nur mit der Default-Output-Routine
}%

\standard \def\signDate{___signdate___}

___content___

\bye
